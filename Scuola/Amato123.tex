\documentclass[11pt]{article}

\usepackage[italian]{babel} 
\usepackage[utf8]{inputenc}

\usepackage{geometry}
\geometry{a4paper}

\usepackage{sectsty}
\allsectionsfont{\sffamily\mdseries\upshape}

\title{Giuliano Amato\\Lezioni dalla crisi\\Rai Edu}
\author{Alessandro Costantini 5 SB}

\begin{document}
\maketitle
\section{Che cos'\`e la crisi}
Il professor Amato inizia la sua lezione parlando della crisi
iniziata il 24 ottobre 1929, che funge da modello per comprendere i
meccanismi che hanno scatenato la crisi attuale.
In particolare si notano le seguenti similitudini:
\begin{itemize}
\item prima dell crollo si \`e verificato un periodo di "follia finanziaria"
durante il quale la finanza ha vissuto di vita propria generando titoli il cui valore reale \`e
sempre pi\`u difficile da quantificare
\item la crisi ha coinvolto direttamente le banche che non possono pi\`u
prestare soldi
\item la crescente difficolt\`a di accesso al credito incide pesantemente sull'economia
reale causando recessione e stagnazione
\end{itemize}

Il presidente J.F. Kennedy introdusse la nozione, ormai divenuta luogo comune,
che in cinese gli ideogrammi della parola crisi possono essere letti
sia come difficolt\`a, sia come opportunit\`a, ma ci\`o che conta \`e 
definire esattamente il termine \emph{crisi}; in particolare bisogna distinguere
tra crisi che investono un'azienda o un luogo particolare \footnote{per esempio il caso di
Palermo che prima della scoperta dell'America era il porto pi\`u importante del mondo
ma con l'apertura delle rotte atlantiche dovette cedere il posto a 
Liverpool che prima era un paese senza grandi prospettive,
o il caso di Aspen, cittadina del Colorado divenuta ricchissima a fine '800 grazie alle miniere d'argento, 
ma che precipit\`o nella crisi pi\`u nera non appena il Congresso decise di usare 
solo l'oro per coniare moneta} e crisi che investono tutto il mondo.\\
Dal punto di vista della moderna economia capitalista le crisi mondiali sono sostanzialmente solo due,
quella del '29 e quella che viviamo oggi, ma va ricordato che
prima della societ\`a industriale l'economia dipendeva dai raccolti
quindi se c'era brutto tempo o una crescita demografica eccessiva si scatenavano
carestie terribili che sono, per l'economia agricola, paragonabili alla crisi economiche mondiali,
poich\'e erano essenzialmente uno squilibrio tra domanda e offerta. 
Il grande riequilibratore di queste crisi agricole, triste a dirsi, erano le grandi pestilenze, che sterminando 
milioni di persone risolvevano e riequilibravano la domanda con l'offerta.\\
L'economia moderna, basata su industria e finanza, \`e un sistema molto pi\`u
complesso specialmente da quando esiste il mercato dei titoli azionari.\\
Nel '800 il ciclo economico seguiva fasi ben precise:
\begin{itemize}
\item fase di espansione della produzione con aumento della ricchezza e dell'emissione di titoli
\item momento in cui la produzione eccede la domanda e quindi i prodotti rimangono invenduti
\item le imprese non hanno capitale sufficiente per pagare i salari, molti operai perdono il lavoro
\item la domanda scende ancora poich\'e molte persone sono disoccupate,
ancora pi\`u prodotti rimangono invenduti e l'emoraggia di capitali impedisce alle aziende di 
onorare i debiti
\item la crisi si sposta al settore finanziario, i titoli e di conseguenza la borsa perdono valore
\item si attuano misure espansive e aiuti alle imprese che ricominciano ad assumere lavoratori
\item il ciclo ricomincia con una nuova fase espansiva
\end{itemize}
Questa catena diventa irresistibile quando con il computer la velocit\`a delle notizie
causa un vero e proprio effetto a cascata, non appena un'impresa \`e in difficolt\`a,
in borsa tende a scatenarsi il panico, tutti gli operatori vendono i titoli che quindi 
perdono ancora pi\`u valore.
Quando la crisi coinvolge una banca si verifica il 
fenomeno dei risparmiatori che vanno a ritirare i soldi allo sportello (spesso restando a mani vuote)
che \`e divenuto il simbolo della crisi mondiale.
Amato ricorda che non essendoci una strategia di programmazione economica
si \`e liberi anche di speculare e commettere errori.
Si passano poi in rassegna le varie teorie che hanno tentato di spiegare le crisi del capitalismo:
\begin{itemize}
\item Malthus: il genere umano dedito al vizio tende a riprodursi pi\`u velocemente della produzione di cibo. 
Questa tesi si \`e dimostrata errata poich\'e non tiene conto della 
tecnologia che ha permesso di sfamare molte pi\`u persone tanto che al giorno d'oggi le carestie sono
dovute alla cattiva distribuzione del cibo e non alla scarsa produzione.
Ciononostante ancora oggi molti vedono nel sovrapopolamento il problema.
\item Marx: il capitalismo segue un ciclo naturale durante il quale:
	\begin{itemize}
	\item la disoccupazione si riduce e i salari salgono, gli operai acquistano pi\`u beni arricchendo i capitalisti
	\item i capitalisti investono in macchine per ridurre il numero di operai
	\item la meccanizzazione crea nuova disoccupazione e riduzione dei salari
	\item il mercato entra in crisi, molte aziende falliscono e vengono acquisite da altre aziende
	\item la riduzione dei salari rende pi\`u facile assumere e il ciclo ricomincia
\end{itemize}
Da notare che secondo Marx questo ciclo non \`e n\'e infinito n\'e sempre uguale,
ma tende ad amplificarsi sempre pi\`u perch\`e le imprese che sopravvivono 
diventano sempre pi\`u potenti (fino al monopolio) e gli operai sempre
pi\`u poveri. Una volta raggiunto il monopolio e lo sfruttamento massimo dei lavoratori, secondo Marx
si dovrebbero verificare i moti rivoluzionari che instaurerebbero un nuovo sistema economico
egalitario (il comunismo). Inoltre Marx sostiene che la crisi \`e insita nel capitalismo stesso, non pu\`o cio\`e
essere considerata un fenomeno eccezionale ma la regola.

\item Keynes: poich\'e l'andamento del mercato finanziario
\`e influenzato dalla speranza nel futuro, le economie di mercato possono attraversare lunghi periodi di 
disoccupazione e recessione durante i quali il pessimismo spinge gli imprenditori (definiti gli \emph{spiriti animali del capitalismo})
a non investire e impedisce ai consumatori di acquistare beni. La ricetta per uscire dalla stagnazione 
\`e un aumento della spesa pubblica per spingere i cittadini-consumatori 
a investire e spendere invece di accumulare ricchezza.
Secondo Keynes l'intervento pubblico nell'economia serve solo a moderare gli effetti
della crisi in modo da evitare rivoluzioni.

\item antikeynesiani: il mercato funziona bene in modo "automatico", le crisi sono causate da effetti esterni al capitalismo,
come l'eccessiva ingerenza dello Stato o di altre istituzioni.
Fra gli antikeynesiani la scuola austriaca \`e la piu liberista e sostiene che le crisi siano 
direttamente causate dall'intervento pubblico che
induce distorsioni nel mercato, creando bolle che poi devono scoppiare. Fra gli austriaci
Schumpeter sostiene che la crisi sia una medicina amara ma utile perch\'e elimina i deboli, come nella selezione naturale.
\end{itemize}

Lasciando da parte idee datate (Malthus) o apertamente antagoniste (Marx), si nota che anche fra gli economisti
favorevoli al capitalismo le opinioni su come affrontare la crisi divergano molto, e ci\`o spiega
come, a seconda del prevalere di una scuola sull'altra, i rimedi contro la crisi siano o siano stati fra i pi\`u incerti e disparati. 

Ancora una volta vale quindi la pena studiare il precedente storico della Grande Depressione del '29
per cercare di capire coma nasce una crisi di portata mondiale.
Si pensi che dopo il crollo di Wall Street gli USA persero il
30\% della loro produzione industriale e il 45\% dei posti di lavoro,
mentre altri paesi come la Germania furono investiti da un'iperinflazione devastante
e destabilizzante finendo ben presto nelle mani di gruppi estremisti che facevano leva sulla disperazione
della gente.\\
Dopo la prima guerra mondiale gli USA vincitori avevano una grande capacit\`a produttiva e 
una notevole liquidit\`a, e anche tra le persone comuni si cominci\`o a capire che i soldi servono 
non solo a finanziare la produzione o acquistare beni, ma anche a fare altri soldi attraverso
gli investimenti finanziari; il problema \`e che la crescita vertiginosa e non regolata
del settore finanziario stimol\`o livelli di produzione impossibili da assorbire
e alcune aziende cominciarono a fallire, i titoli che avevano emesso persero valore e le banche e i cittadini
che possedevano quei titoli si ritrovarono a non aver pi\`u nulla, tanto che
fallirono migliaia di banche e i risparmiatori non riuscirono a ritirare i 
soldi depositati perdendo tutto.

Il rimedio al disastro del '29 fu il New Deal di Roosevelt,
un complesso piano di sostegno pubblico all'economia e di regolamentazione della finanza e dei mercati.

\section{La crisi in America: 2007-2008}
La crisi attuale \`e un'altra grande crisi paragonabile a quella del '29. \`E noto che
l'\emph{annus horribilis} \`e il 2007, ma per individuarne l'origine bisogna 
andare ancora indietro. \\
Come nel '29 la gente si fece possedere dal
\emph{demonietto} della speculazione finanziaria, cos\`i ai giorni nostri
il \emph{demonietto} si \`e diffuso nel mondo dotandosi
di nuovi strumenti fino a diventare un realt\`a
economica nuova capace di generare un flusso di denaro virtuale che
vale decine di volte il flusso delle attivit\`a economiche reali; per esempio
nel 2006 il 40\% dei profitti USA derivavano da attivit\`a finanziarie e il valore giornaliero
degli scambi finanziari in un \emph{singolo giorno} era 60 volte pi\`u grande del valore del commercio 
di beni reali in \emph{un anno}.
Prima della crisi la finanza si era sganciata dall'economia, producendo i suoi titoli e la sua
ricchezza virtuale e generando flussi mai visti in precedenza.
Questa frenesia finanziaria \`e stata resa possibile perch\'e molti paesi hanno consentito che i capitali
si potessero muovere nel mondo con la stessa facilit\`a di beni e servizi.\\
Secondo Jan Kregel, direttore del programma di Politica Monetaria del 
Levy Economics Institute americano, se dopo la crisi del '29 il sistema finanziario era 
stato regolato, dagli anni '90 gli economisti hanno pensato che sarebbe stato reso 
pi\`u efficiente allentando le regole, pertanto il governo elimin\`o gli enti preposti al controllo;
ma come ci dimostra l'attuale tracollo dell'economia mondiale, nessuna di queste idee era giusta.
Infatti nel corso degli anni '90 si sono verificate ripetute crisi finanziarie, e nel 1999 si \`e cercato di stabilizzare
il sistema abrogando il Glass-Steagall Act del 1933 che separava le
banche commerciali da quelle di investimento e creando cos\`i un nuovo tipo di banca, la \emph{holding bancaria},
cio\`e una banca universale che pu\`o fornire qualsiasi servizio
finanziario. Ci\`o permise la nascita
di colossi bancari capaci in soli 10 anni di prendere rischi
eccessivi investendo anche i depositi delle famiglie.
Secondo Kregel questa \`e la causa principale delle difficolt\`a di oggi.

Anche Bertram Schefold professore di  Economia Politica presso l'Universit\`a Goethe di Francoforte
sostiene che Keynes aveva ragione nel considerare  pericolosa una finanza troppo sviluppata, inoltre 
considera necessario un coordinamento internazionale per porre regole chiare al mercato finanziario
poich\'e nessun paese sacrificher\`a il proprio settore finanziario
se gli altri mantengono invariati il loro. La soluzione proposta da Schefold \`e una tassa sulle transazioni
finanziarie da attivare contemporaneamente in tutti i paesi.

In seguito alla crisi del 2007 \`e evidente che ci deve essere un nesso tra finanza (industria del danaro) 
ed economia reale, altrimenti  la disuguaglianza tra reddito massimo e medio-basso 
si fa sempre pi\`u marcata e le sorti del mercato finanziario hanno ripercussioni
eccessive sull'andamento economico.
In particolare, in seguito alla globalizzazione, la finanza ha perso il suo scopo originario di
sostenere la produzione industriale e negli ultimi decenni si \`e assistito al fenomeno dei paesi produttori
che esportano beni e importano titoli di debito pubblico (come la Cina) e altri paesi (come 
gli USA) che sono grandi importatori di beni che pagano esportando titoli.
Tutto ci\`o ha creato un aumento del debito pubblico e un grave squilibrio nella bilancia commerciale mondiale
stimolando finanzieri non sempre onestissimi a creare nuovi strumenti finanziari di complessit\`a crescente.

In seguito all'abrogazione del Glass-Steagall Act nel 1999
sono nati giganti finanziari troppo grandi per essere 
lasciati fallire dallo stato (il famoso principio del \emph{too big to fail})
e che hanno usato i risparmi delle famiglie per sostenere operazioni ad alto rischio;
va ricordato che lo squilibrio della bilancia commerciale mondiale era reso necessario dal debito insostenibile accumulato
dal settore privato e dalle famigie americane e dal debito 
pubblico cresciuto a dismisura a causa delle guerre in Afghanistan e Iraq e anche a causa dello sconto 
fiscale per i redditi pi\`u alti promosso 
da George W. Bush.
Evidentemente se gli USA si indebitavano a prestar loro i soldi non poteva essere che il
resto del mondo e 
infatti gli USA hanno un debito verso Cina e Giappone pari ormai alla loro
intera economia.

Ma come si esercita la fantasia finanziaria?
Chi e perch\'e ha permesso che la finanza arivasse a tanto?
Per arrivare alla situazione attuale non \`e sufficiente aver creduto che i soldi oltre ad essere
investiti in beni possono essere investiti in titoli che producono 
altri soldi (Amato ricorda l'episodio Pinocchio che viene convinto 
dal Gatto e la Volpe a piantare il denaro in un campo per far crescere alberi di soldi), 
ma \`e servita anche una serie di nuovi strumenti che
hanno cambiato i connotati degli operatori finanziari come la banca.
In molti paesi la banca ha smesso di agire in maniera oculata, prestando i soldi a chi poteva restituirli,
e ha aderito al paradigma \emph{originate to distribute}, cio\`e  essere all'origine di un debito
ma distribuire il rischio al di fuori di s\'e, per esempio erogando mutui a soggetti poco affidabili e 
impacchettando i crediti in un nuovo titolo finanziario da vendere sul mercato.
Questo cambiamento nel modo di fare banca ha avuto due effetti:
\begin{itemize}
\item la banca non \`e interessata a valutare il rischio
\item il rischio si diffonde nel mondo con la vana speranza che pi\`u  si diffonde e pi\`u si disperde
\end{itemize}
Ad un certo punto i titoli basati sui mutui ad alto rischio si erano diffusi in tutto il mondo 
e sapendo che comportavano un rischio, si invent\`o 
un altro strumento, il famigerato CDS (Credit Default Swap), ovvero un
titolo che si d\`a in cambio del rischio di morosit\`a 
di chi dovrebbe pagare il titolo principale;
sostanzialmente se chi ha contratto un debito non riesce ad onorarlo,
il titolo di credito emesso dalla banca perderebbe valore quindi la banca
unisce pi\`u titoli di credito in un pacchetto che viene assicurato
contro la morosit\`a dei debitori (se il debitore non paga
interviene la copertura assicurativa). Facendo cos\`i si coinvolgono anche le assicurazioni
che invece dovrebbero avere il portafoglio piu solido. 
Ci sono tutti gli elementi necessari a scatenare un effetto a cascata.
%Jan Kregel ricorda che la crisi ha colto di sorpresa
%sia il governo che gli economisti; si pensava 
%che le posizioni di monopolio di alcune banche ottenute con le 
%regole precedenti (New Deal) avrebbero garantito una 
%certa stabilit\`a al sistema.


\section{Le cause della crisi negli USA}
Pu\`o sembrare assurdo, ma la crisi economica \`e cominciata quando della brava
gente non \`e riuscita pagare la rata del mutuo.
Ovviamente un sistema sano non pu\`o crollare su s\'e stesso per motivi cos\`i
semplici e prevedibili, ma un sistema troppo interconnesso e gonfiato artificialmente
come la finanza internazionale degli inizi del XXI secolo \`e crollato proprio
perch\'e il debito contratto da quei risparmiatori americani si era diffuso e ingigantito
in tutto il mondo e stava alla base di una buona parte della ricchezza virtuale creata in quegli anni.
 
Negli USA per ottenere un mutuo bisognava fornire delle garanzie, ma dalla
seconda met\`a degli anni '90, con lo scopo di garantire una casa per tutti e
per sostenere il settore immobiliare, due grandi agenzie federali hanno garantito per
i pi\`u poveri, inoltre si \`e formato un sistema bancario ombra senza nessun
controllo pubblico e disposto a caricarsi di rischi eccessivi; tutto ci\`o ha creato un gigantesca bolla immobiliare.
La frenesia finanziaria \`e iniziata quando, confidando
nella continua crescita dei prezzi delle case, le holding bancarie
hanno cominciato a concedere mutui anche a chi non aveva sufficienti garanzie, i cosiddetti mutui
\emph{subprime}, per poi impacchettare i crediti in titoli da vendere sul mercato.
Fu proprio la diffusione di questi titoli rischiosissimi a impedire di contenere i danni
quando la bolla immobiliare scoppi\`o nel 2007.
Il meccanismo finanziario usato dalle banche \`e riassumibile in queste fasi:
\begin{itemize}
\item la banca crea una SIV (Special Investment Vehicle), di sua propriet\`a ma
ufficialmente esterna, a cui cede tutti i mutui che ha erogato in cambio di titoli di debito
\item dato che la SIV \`e di propriet\`a della banca, un'eventuale inadempienza da parte di chi ha sottoscritto un
mutuo andrebbe comunque ad influenzare negativamente la banca stessa, quindi la banca unisce i titoli che ha ottenuto dalla SIV
in un pacchetto CDO (Collateralized Debt Obligation) e lo vende sul mercato finanziario, forte anche del fatto che le agenzie di rating 
danno valutazioni AAA a questi titoli
\item le banche che hanno acquistato i CDO decidono di diluire ulteriormente il rischio creando un nuovo strumento, chiamato
CDO\textsuperscript{2}  (CDO al quadrato), che riunisce tanti CDO. A questo punto un CDO\textsuperscript{2} contiene
migliaia di titoli SIV legati alle sorti di migliaia di mutui
\item l'ultimo strumento creato prima del crollo \`e il CDS (Credit Default Swap), ovvero un titolo che deriva il suo valore
dal valore di CDO e CDO\textsuperscript{2}, una sorta di accordo assicurativo secondo il quale l'acquirente
paga il CDS alla banca/assicurazione che lo ha emesso, ma \`e assicurato per una cifra molto superiore
in caso il titolo perdesse di valore.
\end{itemize}
Di fatto tutta questa complessa architettura finanziaria si basava sulla
speranza che il prezzo degli immobili in America sarebbe continuato a salire
indefinitamente. 
Allo scoppio della crisi dei subprime ci si trov\`o
con strumenti estremamente complessi di cui n\'e chi vigilava n\'e chi li
possedeva sapeva quantificare il valore.

Molti si chiedono come sia possibile che nessuno si sia accorto della pericolosit\`a
di questa situazione e a onor del vero analizzando gli 
Atti del Congresso americano, si scopre che nel 2004-2005 molti parlamentari  presentarono
disegni di legge per mettere regole al mercato e verificare i rischi degli ultimi strumenti finanziari,
e molti manager pubblicarono rapporti sui rischi eccessivi presi dalle holding, ma tutti vennero
zittiti. Anzi le agenzie di rating poco prima della crisi diedero una valutazione AAA sia alla holding
bancaria Lehman Brothers sia al colosso assicurativo AIG che fallirono poco dopo
perch\'e erano talmente pieni di titoli tossici che non si riusciva a quantificare attivo e passivo, cosa mai
successa prima.

Eckhard Hein, professore di Economia alla Berlin School of Economics and Law, 
ritiene che quella che stiamo vivendo \`e una crisi del capitalismo fondato sulla finanza, nato negli anni '80
nei paesi anglosassoni (\emph{Reaganomics} e thatcherismo) e caratterizzato da scarsit\`a di regole, 
crescita del settore finanziario
oltre le capacit\`a produttive, eliminazione delle regole del mercato del lavoro 
con conseguente abbassamento dei salari dei
lavoratori e aumento di quelli di manager.
Secondo Hein, bisogna ridurre il settore finanziario e reintrodurre regole per
farlo tornare al suo scopo originario di finanziare le attivit\`a produttive, gli
strumenti pi\`u complessi vanno limitati al massimo, le banche devono
tornare a valutare i debitori, bisogna abolire gli scambi al di fuori
delle borse e dovrebbero essere istituzioni pubbliche, non agenzie di rating private, a valutare
i prodotti finanziari prima di permetterne la vendita, infine sarebbe utile introdurre un'imposta sugli scambi
finanziari, in modo da ridurre gli scambi e provvedere fondi agli stati per
stabilizzare il mercato finanziario.

\end{document}
