\documentclass[11pt]{article}

\usepackage[italian]{babel} 
\usepackage[utf8]{inputenc}

\usepackage{geometry}
\geometry{a4paper}

\usepackage{sectsty}
\allsectionsfont{\sffamily\mdseries\upshape}

\title{Università Ca' Foscari\\Conversazioni sul nostro futuro\\Tito Boeri - Le riforme a costo zero}
\author{Alessandro Costantini 5 SB}

\begin{document}
\maketitle

Tito Boeri, economista, professore all'Università Bocconi di Milano e fondatore della rivista online lavoce.info
espone le tesi contenute nel libro scritto insieme a Pietro Garibaldi 
\textit{"Le riforme a costo zero. Dieci proposte per tornare a crescere"}, Chiare Lettere, 2011.
Il volume presenta dieci proposte che non incidono sul bilancio fiscale del paese ma che
possono contribuire a stimolare la crescita economica in Italia e di conseguenza un miglioramento dei conti pubblici,
risanamento che più volte durante la conferenza viene posto da Boeri come priorità assoluta per il nostro paese
al fine di riacquistare credibilità nei mercati internazionali.
Fin dall'inizio della conferenza Boeri non nasconde che il momento (siamo durante il governo Monti, nell'aprile 2012)
è difficile e delicato, ma sottolinea come il governo Monti sia un
cambiamento positivo rispetto al governo precedente e che va dato merito a Monti di aver intrapreso una serie 
di misure molto incisive per il consolidamento fiscale e per dare maggiore credibilità al paese, come è provato
dalla riduzione dello spread.
Boeri paventa però un'involuzione e peggiori aspettative per il rientro del debito causate da tensioni
nella compagine governativa e dalle critiche rivolte al governo Monti dalle forze politiche
probabilmente a scopo propagandistico-elettorale, sottolineando che invece bisogna evitare di
protestare e far affiorare dubbi sulla durata del governo tecnico per rassicurare i mercati
e non creare ulteriori problemi di credibilità per il paese.
Boeri prosegue la sua analisi sostenendo che il peggioramento della situazione nella primavera del 2012 
è causato da fattori esterni al paese ma aggravato dalle carenze strutturali del sitema
italiano.
La decisione della BCE di aprire una nuova linea di credito a medio termine a tassi
agevolati per le banche europee fornisce però un'occasione importante poiché regala un po' di tempo
per agire al fine di evitare di aumentare le tasse e innescare una spirale recessiva
e l'aumento debito, affrontando invece i problemi delle finanze pubbliche alla radice.
Se il governo italiano ha spesso mancato in questo senso, i governi degli altri paesi in difficoltà non hanno 
di certo brillato nell'affrontare i problemi centrali, come ad esempio il governo Rajoy in Spagna, che non ha 
voluto attaccare il problema del costo eccessivo delle regioni autonome spagnole che creano molto debito; o il
Portogallo che non ha bloccato i costosissimi salvataggi pubblici di grandi imprese
private e ancora l'Irlanda che ha proseguito nei salvataggi dei colossi bancari nazionali.
Insomma un po' tutti non hanno saputo sfruttare la tregua fornita dalla BCE che non potrà piu estendere
le aperture di credito e i mercati di conseguenza cominciano ad innervosirsi.
Secondo Boeri questa mancanza da parte dei partner europei influenza negativamente anche l'Italia, soprattutto considerando che
alcune riforme del governo Monti non sono state efficaci, ad esempio la riforma del mercato del 
lavoro, che avrebbe avuto un importante ruolo nell'aumento della produttività al fine di evitare la stagnazione economica.
Riguardo alla questione del lavoro, il relatore sottolinea come ci sia una disuguaglianza pericolosa
tra una parte di lavoratori che gode di garanzie e protezioni anche eccessive e il mondo del lavoro parasubordinato
che non ha quasi nessuna forma di sostegno né controllo e che purtroppo i giovani appartengono
per la stragrande maggioranza a quest'ultimo settore. Questa disuguaglianza, oltre ad essere ingiusta,
va a minare anche la produttività del lavoro. La riforma Fornero non riesce a unificare le due parti e il passaggio al Parlamento
la indebolirà ancora.
Boeri prosegue poi elencando le sue proposte di riforme a costo zero per l'erario:
\begin{itemize}
\item Politiche dell'immigrazione: quella attuale è miope poiché non valorizza il capitale umano
dei flussi migratori (sono ben note le difficoltà degli stranieri che cercano di venire in italia a fare un dottorato), 
una politica lungimirante punterebbe sul capitale umano e sull'integrazione della seconda generazione di immigrati
\item Apprendistato universitario: il sistema delle lauree 3+2 non
funziona poiché la laurea triennale non ha trovato sbocchi lavorativi e quindi chi si iscrive all'università
è di fatto costretto a conseguire anche la laurea specialistica per avere un titolo spendibile;
quindi va dato valore al triennio con qualifiche professionali che sono molto richieste dal mercato del lavoro.
Il problema delle troppe sedi universitarie di cui molte senza prospettive di crescita e senza la massa critica che serve
per fare ricerca può trasformarsi in una risorsa qualora venissero attivati corsi di apprendistato 
universitario perché potrebbero essere riorientate in senso professionalizzante.
\item Contratto unico a tutele progressive: importante, anche per Spagna e Francia,
sarebbe l'introduzione del contratto unico a tutele progressive e di un salario minimo garantito, che permetterebbero di
abbassare il costo del lavoro e di combattere le sacche di povertà, soprattutto tra i giovani precari.
\item Riforma del pubblico impiego: il tentativo incompiuto di Brunetta è un esempio di riforma sbagliata
perché dava i premi sia a uffici efficienti che a quelli inefficienti e poi obbligava i dirigenti a dare i premi a chi era meritevole. 
Bisogna invece premiare solo le amministrazioni meritevoli e solo dopo far distribuire i premi dai dirigenti.
\item Riforma fiscale per favorire una maggiore partecipazione delle donne al lavoro eliminando le detrazioni per il coniuge a carico
sostituendole con un sussidio condizionato all'impiego
\item Riforma delle libere professioni: qui Monti si e mosso con qualche titubanza, non è stata intaccata la governance
degli ordini che possono dettare legge e invece dovrebbero garantire qualità e rispetto del codice deontologico. Secondo Boeri 
non vanno aboliti gli ordini ma va cambiata la loro governance
\item Riforma delle pensioni: il ministro Fornero ha agito con coraggio ma serviva più flessibilità, così si sarebbe 
evitato il problema degli esodati, inoltre bisogna intervenire con più forza su chi ha pensioni generose
\item Sistema del credito
\item Selezione della classe politica: bisogna intervenire sulla legge elettorale (al tempo il famigerato \textit{Porcellum}, numero
dei parlamentatri e delle cariche; inoltre il livello di istruzione dei parlamentari è molto basso, solo 2/3 sono laureati
\item Diritto di voto esteso ai sedicenni per stimolare la partecipazione dei giovani alla cosa pubblica
\end{itemize}
Boeri conclude facendo notare che \emph{a costo zero} non significa che le riforme da attuare non andranno
ad intaccare gli interessi di chi detiene una qualche forma di potere economico o decisionale (ad esempio le 
lobby), ma è più importante risanare le finanze pubbliche.

\end{document}